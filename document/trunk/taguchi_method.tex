\subsection{Hybrid Orthogonal Crossover}
The previous sections introduced the traditional operations of selection and crossover.
Typically, the introduction of novel genetic diversity through a mutation operation would
immediatly follow crossover. However, this algorithm will employ an additional hybrid
crossover technique in an effort to intelligently create offspring. Techniques from
biological science and process engineering will be implemented to produce the 'most fit'
individual given the available genotype. Specifically, the Taguchi method of orthogonal
array based design of experiments will be shown to radically improve both solution
accuracy and overall convergence speed.

%%%%%%%%%%%%%%%%%%%%%%%%%%%%%%%%%%%%%%%%%%%%%%%%%%%%%%%%%%%%%%%%%%%%%%%%%%%%%%%%
\subsubsection{Design of Experiments: The Taguchi Method}

%%%%%%%%%%%%%%%%%%%%%%%%%%%%%%%%%%%%%%%%%%%%%%%%%%%%%%%%%%%%%%%%%%%%%%%%%%%%%%%%
\subsubsection{Orthogonal Array Generation}

%%%%%%%%%%%%%%%%%%%%%%%%%%%%%%%%%%%%%%%%%%%%%%%%%%%%%%%%%%%%%%%%%%%%%%%%%%%%%%%%
\subsubsection{SNR Calculations}

%%%%%%%%%%%%%%%%%%%%%%%%%%%%%%%%%%%%%%%%%%%%%%%%%%%%%%%%%%%%%%%%%%%%%%%%%%%%%%%%
\subsubsection{Optimal Progeny Generation}