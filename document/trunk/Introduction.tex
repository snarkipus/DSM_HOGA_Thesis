%%%%%%%%%%%%%%%%%%%%%%%%%%%%%%%%%%%%%%%%%%%%%
%
%  CHAPTER 1: Introduction
%
%%%%%%%%%%%%%%%%%%%%%%%%%%%%%%%%%%%%%%%%%%%%%

Mixed-signal systems are systems that possess both analog and digital subsystems. Such
systems are prevalent in test and measurement platforms, data acquisition systems, and
communications devices. Thus, these mixed-signal systems are often central to hardware
applications ranging from common consumer products such as cellular telephones to highly
specialized real-time data collection systems used in mission critical applications such
as space flight.

In mixed-signal systems, the conversion from analog to digital is performed by
an analog-to-digital converter (ADC). Conversely, the conversion from
digital to analog is performed by a digital-to-analog converter (DAC).
These devices are mixed-signal devices that allow for the
ebb and flow of information between the analog world and digital or discrete-time systems
which are now prevalent throughout electrical applications. Because the performance of
digital systems can usually be improved by simple hardware or software changes, the
performance of a mixed-signal system is often limited by the performance of its data
converters.  As a result, the performance of many mixed-signal systems can be improved by
improving the system's data converter performance.

Many different ADC architectures exist and each architecture has its own benefits and
limitations. \DS modulators are an ADC architecture that uses relatively simple analog
circuitry including a low order quantizer and a feedback loop to sample analog signals
with high signal to noise ratios (SNRs) and large dynamic ranges (DRs).  
Because of the simplicity of the architecture, \DS modulators lend themselves to being
implemented in CMOS process technologies which offer mixed-signal electronics, low-power
performance and high levels of integration \cite{kester_analog_2006}.

\DS modulators achieve high SNRs and large DRs by using a feedback loop
filter to attenuate the quantizer's noise in the frequency bands of interest while passing
the input signal to the output. The transfer function describing the loop filter that
attenuates the quantizer's noise is referred to as the $\Delta\Sigma$'s noise transfer
function (NTF). Similarly, the transfer function describing the loop filter that passes
the input signal to the output is called the signal transfer function (STF).  For lowpass
\DS modulators, the NTF is designed as a high-pass filter so that the noise energy is
attenuated within the low-frequency signal band. Conversely, the STF is designed as a
lowpass filter so that the input signals within the low-frequency signal band are not
attenuated. The STF can also act as an anti-aliasing filter. Thus, the output of a \DS
modulator can be modeled as the sum of an input signal filtered by a STF and a noise
source filtered by a NTF.

In this thesis, optimal signal transfer functions (STFs) and noise transfer functions
(NTFs) for \DS modulators are determined using a novel hybrid orthogonal genetic (HOG)
algorithm. For a given oversampling rate (OSR), which is loosely defined as the
ratio of the $\Delta\Sigma$'s sampling frequency to the input signal's Nyquist frequency,
the $\Delta\Sigma$'s STF and NTF are optimized with respect to a weighted combination of
the \DS modulator's signal-to-noise ratio (SNR) and dynamic range (DR).