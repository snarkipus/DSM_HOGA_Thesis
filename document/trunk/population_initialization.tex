\subsection{Population Initialization}
Recall that global optimization algorithms do not require \textit{a priori} knowledge of
the performance surface shape. However, generalized constraints on the objective function
solution space can greatly increase the speed of algorithm convergence. In fact, seeding
the population with known solutions is often employed as a test for conversion latency. It
has also been shown \cite{sarker_evolutionary_2002} that seeding the population with
solutions which are known to be 'good' or 'approximate' can lead to hyper-accurate
solutions.

Recall that both contemporary and classical designs have zero locations which are 
distributed over the operational region along the unit circle in the complex $z$-domain.
It is therefore assumed that the optimal zero locations will be distributed in some
fashion on or near the unit circle as well. The pole locations, being free to exist
outside of the operational region, will assume a shape geometrically commensurate with the
desired frequency response. Thus, we will assume that the pole and zero locations will be
proximal to both the classical and contemporary results. As such, we will seed the initial
population matrix with dithered chromosomes whose undithered coefficients are given by the
classical Chebyshev polynomial based design approach. The added dither will be a zero mean
standard normal random variable with a fixed variance of 10\% (normalized wrt the unit
circle). The \ith\ dither element is expressed as given in the following equation.
%---------------------
\begin{equation}\label{eq:init_dither}
\delta_{i}=\frac{1}{\sqrt{0.2\pi}}e^{-\frac{x^2}{0.2}}
\end{equation}
%---------------------
Given \eqref{eq:init_dither}, we can express the \ith\ member of the initial population as
a dithered chromosome given as follows.
%---------------------
\begin{equation}\label{eq:dithered_chromosome}
\begin{split}
\mathcal{C}_i&=\bigl[
(c_{11}+\delta_{1}),(c_{12}+\delta_{2}),(d_{11}+\delta_{3}),(d_{12}+\delta_{4}),\dotsc, \\
&\qquad(c_{M1}+\delta_{m-4}),(c_{M2}+\delta_{m-3}),
       (d_{M1}+\delta_{m-2}),(d_{M2}+\delta_{m-1}),(k+\delta_{m}) \bigr]\\
&=\bigl[
\tilde{c}_{11},\tilde{c}_{12},\tilde{d}_{11},\tilde{d}_{12},\tilde{c}_{21},\tilde{c}_{22},
\tilde{d}_{21},\tilde{d}_{22},\dotsc,\tilde{c}_{M1},\tilde{c}_{M2},\tilde{d}_{M1},
\tilde{d}_{M2},\tilde{k}\bigr]
\end{split}
\end{equation}
%---------------------
The initial population is a $m\times n$ matrix whose $n$ columns are the dithered
chromosomes as given in \eqref{eq:init_dither} and $m$ rows correspond to the elements of
the chromosomes. This population matrix, denoted $\mathbb{G}$, is given by the following
equation where $T$ indicates the transpose operation.
\vspace{1mm}
\begin{equation}
\mathbb{G}_{\text{init}} =\bigl[\mathcal{C}_1^T \mathcal{C}_2^T \dotsb
\mathcal{C}_n^T\bigr]=
\begin{pmatrix}
\tilde{a}_{11,1}&\tilde{a}_{11,2}&\hdotsfor{5}&\tilde{a}_{11,n-1}&\tilde{a}_{11,n} \\
\tilde{b}_{12,1}&\tilde{b}_{12,2}&\hdotsfor{5}&\tilde{b}_{12,n-1}&\tilde{b}_{12,n} \\
\tilde{c}_{11,1}&\tilde{c}_{11,2}&\hdotsfor{5}&\tilde{c}_{11,n-1}&\tilde{c}_{11,n} \\
\tilde{c}_{12,1}&\tilde{c}_{12,2}&\hdotsfor{5}&\tilde{c}_{12,n-1}&\tilde{c}_{12,n} \\
\tilde{d}_{11,1}&\tilde{d}_{11,2}&\hdotsfor{5}&\tilde{d}_{11,n-1}&\tilde{d}_{11,n} \\
\tilde{d}_{12,1}&\tilde{d}_{12,2}&\hdotsfor{5}&\tilde{d}_{12,n-1}&\tilde{d}_{12,n} \\
\vdots&\vdots&&&\ddots&&&\vdots&\vdots						   \\
\tilde{c}_{M1,1}&\tilde{c}_{M1,2}&\hdotsfor{5}&\tilde{c}_{M1,n-1}&\tilde{c}_{M1,n} \\
\tilde{c}_{M2,1}&\tilde{c}_{M2,2}&\hdotsfor{5}&\tilde{c}_{M2,n-1}&\tilde{c}_{M2,n} \\
\tilde{d}_{M1,1}&\tilde{d}_{M1,2}&\hdotsfor{5}&\tilde{d}_{M1,n-1}&\tilde{d}_{M1,n} \\
\tilde{d}_{M2,1}&\tilde{d}_{M2,2}&\hdotsfor{5}&\tilde{d}_{M2,n-1}&\tilde{d}_{M2,n} \\
\tilde{k}_{1}&\tilde{k}_{2}&\hdotsfor{5}&\tilde{k}_{n-1}&\tilde{k}_{n} \\
	
\end{pmatrix}
\end{equation}
% \vspace{1mm}

The appropriate sizing for a population is directly related to the complexity of the
problem space. However, this relationship can be difficult, if not impossible, to
adequately characterize \cite{alander_optimal_1992}. Additionally, the relative size of
the population also dictates the nature of the results. For instance, a large
unconstrained population will offer a broad evaluation of the performance surface at the
expense of convergence speed and steady-state misadjustment. In contrast, a smaller
constrained population will converge more quickly at the expense of increased result
diversity.

For generally contrained problem spaces, perhaps the most important concept is minimum
population size. Because minimum population size is tightly coupled with the problem space
complexity, parameters such as this are typically refined or 'tuned' \textit{a posteriori}
\cite{schaffer_-_1989},\cite{alander_optimal_1992}. However, it is also plausible to use
known design parameters applications with similar problem space complexity. Thus, the
initial size of the population has been restricted to 200 for this application and has
proven sufficient for all cases. Note that there are diminishing returns when the
population size becomes prohibitively large. Greatly increasing the population size is
computationally intensive making convergence times unacceptable for most applications.