\subsection{Chromosome Representation}
For every generation, each member of the population is fully characterized by its
chromosome. Specifically, the formulation of this chromosome contains the genetic
information available for reproduction. During reproduction, both traditional and hybrid,
the genetic information contained in the chromosome is exchanged between the progenitors.
This exchange is discussed at length in the following sections.

Structurally, chromosomes are implemented as linear arrays of length $m$. These arrays are
populated with $m$ traits, or alleles, that are iteratively evolved during the
evolutionary process. For this application, the alleles correspond with the coefficients
of the IIR filters which govern the \DSm functionality. Recall that the NTF (and STF) are
designed as IIR filters of the form given in the following equation.
%---------------------
\begin{equation}\label{eq:chromosome_NTF_1}
\textrm{NTF}(z)=\frac{\displaystyle\sum_{i=0}^{M}{a_i
z^{-i}}}{\displaystyle\sum_{i=0}^{M}{b_i z^{-i}}}				
=\frac{\displaystyle\prod_{j=0}^{N}{\bigl(1-c_j z^{-1}\bigr)}}				
		{\displaystyle\prod_{j=0}^{N}{\bigl(1-d_j z^{-1}\bigr)}} 		
\end{equation}
%--------------------- 
Either the polynomial or pole-zero representations shown in \eqref{eq:chromosome_NTF_1}
could be directly implemented in chromosomes where the alleles represent the coefficients.
However, storing the unfactored polynomial coefficients yields chromosomes which are
overly sensitive to perturbation during the evolutionary process. Secondly, implementing
only first order poles and zeros forces the algorithm to directly account for pole-zero
conjugation which significantly lengthens run times. Thus, to reduce coefficient
sensitivity and foster implied conjugation, the chromosomes will be of the following form
\cite{kit-sang_tang_design_1998}.
%---------------------
\begin{equation}\label{eq:chromosome_NTF_2}
\textrm{NTF}(z)=k\displaystyle\prod_{i=0}^{N}\frac{\bigl(1-a_i z^{-1}\bigr)}{\bigl(1-b_i
z^{-1}\bigr)}							
\displaystyle\prod_{j=0}^{M}\frac{\bigl(1+c_{1j}z^{-1}+c_{2j}z^{-2}\bigr)}{\bigl(1+d_{1j}
z^{-1}+d_{2j}z^{-2}\bigr)}	   
\end{equation}
%--------------------- 
Note that one zero-pole pair will exist singularly on the real axis for odd ordered
systems. Thus, $N$ in \eqref{eq:chromosome_NTF_2} is either 1 or 0 for odd or even system
order respectively. Subsequently, the order of the system is then given by the following
equation.
%---------------------
\begin{equation}\label{eq:filter_order}
\text{ORDER}=2M+N	   
\end{equation}
%--------------------- 

The ones are omitted and only the coefficients are stored in the chromosome. As such,
the encoding for even and odd order functions is given by the following expressions where
$\mathcal{C}$ denotes the chromosome.
%---------------------
\begin{subequations}	
\begin{align}
%---------------------
\mathcal{C}_\text{even}&=\bigl[c_{11},c_{12},d_{11},d_{12},c_{21},c_{22},d_{21},d_{22},
\dotsc,c_{M1},c_{M2},d_{M1},d_{M2},k\bigr]\\
\mathcal{C}_\text{odd}&=\bigl[a_1,b_1,c_{11},c_{12},d_{11},d_{12},c_{21},c_{22},d_{21},
d_{22},\dotsc,c_{M1},c_{M2},d_{M1},d_{M2},k\bigr]
%---------------------
\end{align}
\end{subequations}
%---------------------

Note that the length, $m$, of the chromosomes corresponds to the dimension of the
problem space. The system order will be restricted to 8 or less. Thus, the dimension of
the problem space is limited to 33 or less. The proposed algorithm has been tested on
problem dimensions up to 100 and is thus well suited for this application.