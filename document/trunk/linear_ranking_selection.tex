\subsection{Linear-Ranking Selection}
As mentioned previously, 'more fit' individuals have a higher likelyhood of reproducing
than 'less fit' individuals. The mechanism which affects this fitness-based
genetic discrimination is referred to as selection. The operating principle for
selection is that a chromosome's probability for mating eligibility is a function of its
fitness with respect to the greater population.

Pure random selection methods suffer from stochastic variability in that there is no
guarantee that all members will be considered for selection. Thus, the possibility exists
that good chromosomes may be inadvertently discarded. Additionally, being application
agnostic, the algorithm assumes that the qualitative 'good vs. bad' indicated by a
particular range of fitness values is sufficient. In practice, however, population
dynamics exist such that the difference between 'good' and 'bad' is poorly reflected in
the observed fitness even for well formed objective functions
\cite{blickle_comparison_1996}.

One method of countering this phenomenon is to implement an 'elitist'
algorithm. By definition, the best chromosome(s) are placed into the next
generation and as well as selected for mating eligibility. This technique has
been shown to greatly increase convergence speeds, especially when steady-state
misadjustment is significant\cite{reeves_genetic_2002}.

Another means of improving selection efficacy is to implement a ranking scheme by
which the population is ordered with respect to its fitness. This ordering removes any
unintended statistical bias from an ill formed objective function. The HOES is a
minimization based evolutionary strategy, and thus the term 'best' corresponds with the
smallest cost. As such, the individuals of a population are sorted according to their
fitness values and the rank 1 is assigned to the worst (highest cost). Similarly, the
rank $N$ is assigned to the best individual (lowest cost) for a population of size $N$.
The probability of selection for mating eligibility is then linearly assigned according
to the individual's rank within the greater population. This is given by the following
equation where $\alpha$ and $\beta$ are positive scalars.
%---------------------
\begin{equation}\label{eq:linear_ranking}
%p_{i}=\frac{1}{N}\left(\eta^{-}+\bigl(\eta^{+}-\eta^{-}\bigr)\frac{i-1}{N-1}\right);
%\quad i\in\{1,\dotsc,N\}
p_{i} = \alpha + \beta i
\end{equation}
%---------------------
Note that $p_{i}$ is a probability and therefore subject to its distribution. Thus,
we can show the following.
%---------------------
\begin{equation}\label{eq:linear_ranking_pdf}
\sum_{i=1}^{N}\bigl(\alpha+\beta i\bigr) = 1
\end{equation}
%---------------------
Solving the summation, we arrive at the following expression.
%---------------------
\begin{equation}\label{eq:linear_ranking_pdf2}
N\left(\alpha + \beta \frac{N+1}{2}\right)=1
\end{equation}
%---------------------

Generally, selection pressure can be defined as the degree of emphasis that is placed on
selecting better suited individuals over worse ones within a given population. As done
here \cite{reeves_genetic_2002}, the selection pressure will be denoted as
$\phi_{x}$ and is described by the following equation.
%---------------------
\begin{equation}\label{eq:linear_ranking_selection}
\phi_{\text{max}} = \frac{p_{\text{max}}}{p_{\text{median}}}
\end{equation}
%---------------------
Substituting \eqref{eq:linear_ranking_pdf2} into \eqref{eq:linear_ranking_selection} we
arrive at the following equation.
%---------------------
\begin{equation}\label{eq:linear_ranking_selection_2}
\phi_{\text{max}} = \frac{\alpha+\beta N}{\alpha+\beta(N+1)/2}
\end{equation}
%---------------------
Finally, solving \eqref{eq:linear_ranking_selection_2} for $\alpha$ and $\beta$ yields
the following expressions.
%---------------------
\begin{subequations}\label{eq:linear_ranking_selection_3}
\begin{align}
\alpha &= \frac{2N-\phi_{\text{max}}\bigl(N+1\bigr)}{N\bigl(N-1\bigr)} \\
\beta  &= \frac{2(\phi_{\text{max}}-1)}{N\bigl(N-1\bigr)}
\end{align}
\end{subequations}
%---------------------
Because $\alpha$ and $\beta$ are positive scalars, it is clear from
\eqref{eq:linear_ranking_selection_3} that $1\leq\phi_{\text{max}}\leq2$. Note that
regardless of individual fitness, the 'most fit' individual is no more than twice as
likely than average to be selected for mating eligibility. A similar analysis can be made
for the relative probability for selection of the 'least fit' individual, denoted as
$\phi_{\text{min}}$.In summary, we can show the following
regarding selection pressure.
%---------------------
\begin{subequations}\label{eq:linear_ranking_phi}
\begin{align}
\phi_{\text{max}} &= 2 - \phi_{\text{min}} \\
\phi_{\text{min}} \geq 0 &\qquad 1\leq\phi_{\text{max}}\leq2
\end{align}
\end{subequations}
%---------------------

In practice, the selection pressure will be fixed such that $\phi_{\text{max}}$
is 1.1\cite{back_optimization_1991}. This provides an adequate balance between exploration
and exploitation of the performance surface. Subsequent to selection, all eligible
chromosomes participate in a stochastic exchange of alleles referred to as crossover.
This will be addressed in the following section.