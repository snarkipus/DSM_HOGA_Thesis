%%%%%%%%%%%%%%%%%%%%%%%%%%%%%%%%%%%%%%%%%%%%%%%%%%%%%%%%%%%%%%%%%%%%%%%%%%%%%%%%
%
% 2.2.2: Zero-Optimized Design (aka Schreier's method)
%
%%%%%%%%%%%%%%%%%%%%%%%%%%%%%%%%%%%%%%%%%%%%%%%%%%%%%%%%%%%%%%%%%%%%%%%%%%%%%%%%


%%%%%%%%%%%%%%%%%%%%%%%%%%%%%%%%%%%%%%%%%%%%%%%%%%%%%%%%%%%%%%%%%%%%%%%%%%%%%%%%
\subsection{Zero Optimized Design}
As illustrated in the preceding section, classical design methodologies are iterative 
processes requiring significant general knowledge in \DSm design practices. The 
designer is challenged with trying to balance maximal suppression of the inband noise 
with the finite limitations of greater system stability. Thankfully, countless design
 examples exist in literature and are continually updated as new process 
technologies are exploited.

While classical methods are sufficient for practical purposes, they only produce 
systems which meet or exceed some arbitrary set of design requirements. They do not 
provide an ideal system for any particular set of requirements. As such, Richard 
Schreier developed a design methodology  \cite{schreier_understanding_2004} and 
supporting MATLAB toolset which greatly reduces the academic overhead for \DSm 
system designers.

It is Schreier's belief that NTF performance is largely dictated by its out-of-band gain 
(OOBG) and its zero locations with the pole distribution making only a secondary 
contribution. Thus, his algorithm iterates the pole locations according to fixed, 
optimized zero locations and out-of-band gain constraints as illustrated in the 
following sections.

%%%%%%%%%%%%%%%%%%%%%%%%%%%%%%%%%%%%%%%%%%%%%%%%%%%%%%%%%%%%%%%%%%%%%%%%%%%%%%%%
\subsubsection{Optimal Zeros and Noise Distribution}
When compared against NTFs of the form given in \eqref{eq:schreier_NTF_form}, a 
significant improvement in SQNR can be achieved by redistributing the zeros uniformly 
across the operational region of the modulator. 
%-------------------
\begin{equation}\label{eq:schreier_NTF_form}
 \text{NTF}_{\text{simple}}=\frac{\bigl(1-z^{-1}\bigr)^N}{A(z)}
\end{equation}
%-------------------
Specifically, placing the zeros at $m$ equidistant points on the unit circle allows 
for uniform attenuation of the noise within the operational region.

For NTFs of the form given in \eqref{eq:schreier_NTF_form}, we can say the following 
regarding magnitude.
%-------------------
\begin{equation}\label{eq:schreier_zero_opt}
\mag{\text{NTF}}\approx\frac{\omega^{n}}{\mag{A(1)}}
\end{equation}
%-------------------


%%%%%%%%%%%%%%%%%%%%%%%%%%%%%%%%%%%%%%%%%%%%%%%%%%%%%%%%%%%%%%%%%%%%%%%%%%%%%%%%
\subsubsection{Stability Criterion}

%%%%%%%%%%%%%%%%%%%%%%%%%%%%%%%%%%%%%%%%%%%%%%%%%%%%%%%%%%%%%%%%%%%%%%%%%%%%%%%%
\subsubsection{DelSig Toolbox (MATLAB)}